% MOL518 Homework 1
% (Edit due date, points, and any course policy text as needed.)

\documentclass[11pt]{article}

\usepackage[margin=1in]{geometry}
\usepackage{amsmath, amssymb}
\usepackage{graphicx}
\usepackage{hyperref}
\usepackage{enumitem}
\usepackage{float}
\usepackage{booktabs}
\usepackage{xcolor}
\usepackage{listings}

\hypersetup{colorlinks=true, linkcolor=blue, urlcolor=blue, citecolor=blue}

% Simple code listing style
\lstset{
  basicstyle=\ttfamily\small,
  frame=single,
  breaklines=true,
  showstringspaces=false,
  keywordstyle=\color{blue},
  commentstyle=\color{gray},
  stringstyle=\color{teal}
}

\newcommand{\course}{MOL518: Intro to Data Analysis}
\newcommand{\hwnum}{Homework 1}
\newcommand{\duedate}{\textbf{Due:} \underline{\hspace{2.5in}}}

\begin{document}

\begin{center}
{\Large \course}\\[2pt]
{\large \hwnum}\\[6pt]
\duedate\\[10pt]
\end{center}

\noindent\textbf{Name:} \underline{\hspace{2.5in}} \hfill \textbf{NetID:} \underline{\hspace{1.5in}}\\
\textbf{Section/Preceptor (if applicable):} \underline{\hspace{2.2in}}

\vspace{10pt}

\section*{Instructions}
\begin{itemize}[leftmargin=*]
  \item Submit a single PDF generated from this \LaTeX{} file.
  \item Show your work clearly. When you write code, include the exact code you ran (copy/paste into a \texttt{lstlisting} block) and the relevant output.
  \item Unless stated otherwise, you may use standard Python libraries covered in Lecture 1--2: built-ins, NumPy, and basic plotting tools if introduced in your section.
  \item If you use AI tools, follow the course AI-use policy and cite what you used and how (briefly).
\end{itemize}

\section*{Problem 1: Thinking Like a Computer (Algorithms)}
(10 points)

\begin{enumerate}[label=\alph*)]
  \item Write a step-by-step algorithm (plain English) for computing the mean (average) of a list of numbers.
  \item Write a step-by-step algorithm (plain English) for finding the maximum value in a 1D array, without using any built-in \texttt{max} or NumPy convenience functions.
\end{enumerate}

\section*{Problem 2: Variables, Types, and Common Errors}
(15 points)

\begin{enumerate}[label=\alph*)]
  \item Explain the difference between the number \texttt{5} and the string \texttt{"5"}. Give one example of an operation that behaves differently for each.
  \item The code below throws an error. Rewrite the code so it runs, and briefly explain what was wrong.

\begin{lstlisting}[language=Python]
number = 5
text = "5"
print(text + number)
\end{lstlisting}

  \item The code below throws an error. Rewrite the code so it runs, and briefly explain what was wrong.

\begin{lstlisting}[language=Python]
radius = 2
pi = 3.14159
surface_area = 4 * Pi * radius**2
print("The sphere surface area is", surface_area)
\end{lstlisting}

\end{enumerate}

\section*{Problem 3: NumPy Arrays (Indexing, Slicing, Shape)}
(25 points)

\begin{enumerate}[label=\alph*)]
  \item Create a 1D NumPy array containing the odd integers from 1 to 21 (inclusive).
  \item Print every 3rd element of that array (starting from the first element of the array).
  \item How many elements remain if you remove every 3rd element? Show your reasoning and/or code.
  \item Create the following 2D array as a NumPy array:

\begin{center}
\begin{tabular}{ccccc}
\toprule
0 & 0 & 1 & 0 & 0\\
0 & 2 & 0 & 2 & 0\\
3 & 0 & 0 & 0 & 3\\
0 & 2 & 0 & 2 & 0\\
0 & 0 & 1 & 0 & 0\\
\bottomrule
\end{tabular}
\end{center}

  \item Print the 3rd row (be careful about Python indexing).
  \item Compute the total number of elements in the array using its \texttt{.shape}.
  \item Replace the last row with \texttt{[99, 99, 99, 99, 99]} and print the modified array.
\end{enumerate}

\section*{Problem 4: Loading and Inspecting Data (CSV)}
(25 points)

In Lecture 2 you loaded a growth curve CSV file into a NumPy array.

\begin{enumerate}[label=\alph*)]
  \item Load \texttt{Lecture\_2/data/growth\_curve1.csv} using \texttt{np.loadtxt(..., delimiter=",")}. Include the code you used.
  \item Print the shape of the loaded array.
  \item Print the first 5 rows and the last row.
  \item Extract the time column and OD column into 1D arrays called \texttt{time} and \texttt{od}.
  \item Compute the total duration of the experiment in hours.
\end{enumerate}

\section*{Problem 5: Saving Derived Data}
(15 points)

\begin{enumerate}[label=\alph*)]
  \item Create a normalized OD array \texttt{od\_norm} by dividing all OD values by the first OD measurement.
  \item Create a 2D array with time in hours and \texttt{od\_norm} as columns.
  \item Save the result as a new CSV file (do \emph{not} overwrite the original). Name it \texttt{growth\_curve1\_hr\_norm.csv} and place it alongside the original in the same \texttt{data/} directory.
\end{enumerate}

\section*{Optional (Extra Credit): Two Files, Lists, and Change in OD}
(+5 points)

Load both \texttt{growth\_curve1.csv} and \texttt{growth\_curve2.csv} using a Python list of filenames. For each dataset, compute the change in OD from the first to last time point. Print the two values.

\vspace{8pt}
\noindent\textbf{Checklist before submitting:}
\begin{itemize}[leftmargin=*]
  \item PDF compiles without errors.
  \item All questions answered and labeled.
  \item Code included where requested.
  \item Output/units reported where relevant.
\end{itemize}

\end{document}
