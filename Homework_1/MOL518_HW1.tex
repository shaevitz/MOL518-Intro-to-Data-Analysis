\documentclass[11pt]{article}

\usepackage[margin=1in]{geometry}
\usepackage{amsmath, amssymb}
\usepackage{graphicx}
\usepackage{hyperref}
\usepackage{enumitem}
\usepackage{float}
\usepackage{booktabs}
\usepackage{xcolor}
\usepackage{listings}

\hypersetup{colorlinks=true, linkcolor=blue, urlcolor=blue, citecolor=blue}

% Simple code listing style
\lstset{
  basicstyle=\ttfamily\small,
  frame=single,
  breaklines=true,
  showstringspaces=false,
  keywordstyle=\color{blue},
  commentstyle=\color{gray},
  stringstyle=\color{teal}
}

\newcommand{\course}{MOL518 Spring 2026}
\newcommand{\hwnum}{Homework 1}
\newcommand{\duedate}{\textbf{Due: Friday February 6, 2026}}

\begin{document}

\begin{center}
{\Large \course}\\[2pt]
{\large \hwnum}\\[6pt]
\duedate\\[10pt]
\end{center}


For this assignment, you will generate and then submit a Jupyter notebook on Canvas. You may use generative AI tools as per the syllabus unless otherwise stated in the problem.

\vskip 4pt

\noindent \textit{First, make a new Jupyter notebook named \texttt{MOL518\_HW1\_<YourLastName>.ipynb}. All problems will be answered in this notebook. For each problem, start with a Markdown cell using Problem XX as the title, i.e. ``\# Problem 1''.}

\newpage 

\section*{Problem 1: Practice with Markdown}

Figure~\ref{fig:stryerpreface} is the preface page for Lubert Stryer's classic textbook \emph{Biochemistry}. Recreate this page in a Markdown cell in your notebook. The image at the bottom of the page is provided as \texttt{StryerPreface\_liverketone.png}. {\it Do not use AI for this problem.}

\begin{figure}[H]
\centering
\includegraphics[width=0.6\textwidth]{StryerPreface.png}
\caption{Stryer preface page.}
\label{fig:stryerpreface}
\end{figure}

\section*{Problem 2: Wet-Lab Arithmetic}

You are doing a phage infection assay. You have a tube of virus stock at a known titer, and you want to prepare an inoculum at a target multiplicity of infection (MOI) in a specified final volume. You will now prepare a mini ``protocol note'' for yourself: define variables clearly, keep track of units, and print a short summary at the end.

\begin{enumerate}[label=\alph*)]
  \item In a single code cell, define the following variables:
  \begin{itemize}
    \item \texttt{cells} = 250000 \quad (cells)
    \item \texttt{moi} = 1.5 \quad (dimensionless)
    \item \texttt{titer} = 2.0e8 \quad (infectious units per mL)
    \item \texttt{inoc\_vol\_uL} = 500 \quad ($\mu$L total inoculation volume)
  \end{itemize}

  \item Compute the number of infectious units needed for this infection, \texttt{iu\_needed}. Briefly explain what MOI means in words.

  \item Convert the titer to infectious units per $\mu$L, then compute the volume of virus stock to add (in $\mu$L) to deliver \texttt{iu\_needed}.
  
  \item Compute the remaining volume (in $\mu$L) that must be filled with media so the final inoculum volume is exactly \texttt{inoc\_vol\_uL}.

  \item Print a short summary that includes: \texttt{cells}, \texttt{moi}, \texttt{iu\_needed}, virus volume ($\mu$L), media volume ($\mu$L).
\end{enumerate}

\section*{Problem 3: CSV Sanity Check}

You ran a plate reader growth experiment and saved the results as a CSV file. Before doing any serious analysis, you want to do a fast sanity check to confirm the file loaded correctly, the sampling looks consistent, and the values are in a reasonable range.

Your goal is to produce a short report in your notebook: show what you loaded, a few key numbers, and make it easy for a reader to understand what the dataset looks like.

\begin{enumerate}[label=\alph*)]
  \item Load \texttt{Lecture\_2/data/growth\_curve1.csv} as a 2D NumPy array called \texttt{data}.
  \item Print the shape of \texttt{data}. Then print the first row and the last row, and briefly describe what those rows mean.
  \item Extract \texttt{time} in seconds and \texttt{od} as 1D arrays.
  \item Compute and print the sampling interval in seconds, i.e. the time between sequential timepoints.
  \item Compute and print the total duration of the experiment in hours.
  \item Compute and print the following summary quantities, and label each one clearly in the output:
  \begin{itemize}
    \item absolute OD change
    \item total fold change in OD
  \end{itemize}
\end{enumerate}

\section*{Problem 4: Create, Transform, and Save a Synthetic Dataset (No Loops)}
(15 points)

In real life, your first dataset might be messy: missing points, noise, odd formatting, or surprising values. In this problem, you will generate a tiny \emph{synthetic} dataset where you control the structure and the meaning of each column.

Think of this as a practice run for the workflow: create arrays, combine them into a table, write to disk, reload, and confirm nothing got scrambled.

\textbf{Do not use \texttt{for} loops or \texttt{if} statements in this problem.}

\begin{enumerate}[label=\alph*)]
  \item Create a NumPy array called \texttt{time\_hr} with these time points (hours): \texttt{[0, 1, 2, 4, 8, 16]}. Print it.
  \item Define \texttt{od0 = 0.08} and \texttt{doubling\_time\_hr = 1.5}. In one Markdown sentence, explain (conceptually) what ``doubling time'' means.
  \item Compute \texttt{od} using: \texttt{od = od0 * 2**(time\_hr/doubling\_time\_hr)}. Print \texttt{od}.
  \item Create \texttt{od\_norm} by normalizing \texttt{od} to its first value. Print \texttt{od\_norm}.
  \item Use \texttt{np.column\_stack} to make a 2D array (a mini table) with columns \texttt{time\_hr}, \texttt{od}, \texttt{od\_norm}. Print the resulting array and its shape.
  \item Save the result to \texttt{Homework\_1/simulated\_growth.csv} using \texttt{np.savetxt(..., delimiter=",")}. (This path is relative to the repository root.)
  \item Reload the file you saved into \texttt{data2}. Print \texttt{data2.shape} and the first 3 rows, and confirm (in a sentence) that the reloaded numbers match what you expect.
\end{enumerate}

\section*{Problem 5: Lists --- Sample Tracking and Lane Batching (No Loops)}
(15 points)

You are organizing samples for a downstream assay (e.g., loading lanes on a gel, assigning sequencing lanes, or planning a plate layout). In practice, a lot of mistakes come from simple bookkeeping: missing a sample, duplicating a label, or mixing the order.

In this problem, you will treat a Python list as your ``sample sheet'' and practice simple, reliable operations: counting, appending, sorting, and slicing into batches.

\textbf{Do not use \texttt{for} loops or \texttt{if} statements in this problem.}

\begin{enumerate}[label=\alph*)]
  \item Create a Python list called \texttt{samples} with the following sample IDs (strings), in this exact order:

\begin{lstlisting}[language=Python]
samples = [
    "WT_A", "WT_B", "WT_C", "KO1_A", "KO1_B", "KO1_C",
    "KO2_A", "KO2_B", "KO2_C", "Rescue_A", "Rescue_B", "Rescue_C"
]
\end{lstlisting}

  \item Print the number of samples using \texttt{len(samples)} and briefly interpret what that number means in this context.
  \item Append two new samples to the end of the list: \texttt{"Blank"} and \texttt{"PositiveCtrl"}. Print the updated list.
  \item Create a new list called \texttt{samples\_sorted} that is an alphabetically sorted version of \texttt{samples} (do not change the original list). Print both lists to demonstrate that only the new one was sorted.
  \item Imagine you can only process 5 samples per ``lane'' (or batch). Create three new lists by slicing \texttt{samples\_sorted}:
  \begin{itemize}
    \item \texttt{lane1} = first 5 samples
    \item \texttt{lane2} = next 5 samples
    \item \texttt{lane3} = remaining samples
  \end{itemize}
  \item Print \texttt{lane1}, \texttt{lane2}, and \texttt{lane3}. In a short Markdown note, explain why slicing is a safer approach than manually retyping sample IDs.
\end{enumerate}




\end{document}
